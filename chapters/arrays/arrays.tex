%================================================
% CAPÍTULO ARRAYS
%================================================

\chapterimage{chapter_pointers.jpg} % Chapter heading image

\chapter{Arrays}
Quem nunca precisou organizar dados de tipo semelhante num único local? Eu sim,
você não? Então quando você, por exemplo, escreve um texto num pedaço de papel,
o que acha que está fazendo?  \say{\textit{Organizando ideias?}}. Sim, também!
Contudo, o que você está organizando materialmente são caracteres.

Imagine agora que, usando o mesmo texto do papel no computador, para cada
caractere organizado tenhamos que definir uma variável do tipo \textit{char} e
atribuir-lhe o valor respectivo.

\begin{ccode}
  char c1, c2, c3, c4, c5, c6, c7, c8, c9, c10,
       c11, c12, c13, c14, c15, c16, c17;

  c1 = 'A';
  c2 = 'u';
  c3 = 't';
  c4 = 'h';
  c5 = 'o';
  c6 = 'r';
  c7 = ' ';
  ...
\end{ccode}

Tal exemplo é deveras trabalho do capiroto, e só será utilizado na vida real se
o programador tiver algum pacto com esse tinhoso. Para os que seguem o caminho
da luz e dos ensinamentos jedi há uma alternativa feliz.

\textbf{Arrays}, em C, são tipos de dados de alocação contíngua derivados dos
tipos simples. Em outras palavras, são estruturas de dados homogêneas,
compostas de elementos de mesmo tipo, e posicionados em sequência na
memória. Arrays também são conhecidos como \textit{vetores}, quando a sequência
dos dados é unidimensional (linear), e como \textit{matrizes}, quando essa
sequência se dá multimensionalmente (linhas e colunas). Divagaremos a respeito
dos arrays do tipo matriz mais a frente, abordando, por enquanto, o array do
tipo vetor. Seguem dois exemplos de arrays.

\begin {center}
  \begin{bytefield}[bitwidth=1em]{12}
    \bitbox{12}{Smug Author!}\\
    \bitbox{1}{S} \bitbox{1}{m} \bitbox{1}{u} \bitbox{1}{g} \bitbox{1}{ }
    \bitbox{1}{A} \bitbox{1}{u} \bitbox{1}{t} \bitbox{1}{h} \bitbox{1}{o}
    \bitbox{1}{r} \bitbox{1}{!}
  \end{bytefield}
\end{center}

Nesse primeiro, a frase \say{Smug Author!} é a organização de doze caracteres
um atrás do outro. Isso mesmo, doze, pois o espaço também é um dado. Nunca se
esqueça que, quando tratamos dados, até um \say{0} ocupa espaço na memória do
computador.

Abaixo outra organização de dados. Desta vez de inteiros que serão o resultado
do próximo sorteio da mega-sena. Duvida?

\begin{center}
  \begin{bytefield}[endianness=little,bitwidth=6em]{6}
    \bitbox{1}{9} \bitbox{1}{20} \bitbox{1}{11}
    \bitbox{1}{7} \bitbox{1}{43} \bitbox{1}{31}\\
    % \bitheader{0-5}
  \end{bytefield}
\end{center}

\begin{remark}
  Em algumas fontes esparsas em língua portuguesa, você, leitor, encontrará
  referências a arrays como sendo \textit{\textbf{arranjos}}. Ressaltamos que
  um array, numa tradução mais fiel, é um \textit{\textbf{conjunto}} de
  dados. Já um arranjo:
  \begin{figure}[!htp]
    \centering
    \includegraphics[scale=0.8, keepaspectratio=true]{arrays_arrange.jpg}
    \caption{Array não é Arranjo}
    \label{fig:arrays_arrange}
  \end{figure}
\end{remark}

\section{Usando Arrays}
\subsection{Definindo Arrays}

Ao definirmos um array, precisamos determinar a respectiva quantidade de
elementos.

\begin{ccode}
  int i[6];
  char c[3];
\end{ccode}

Os colchetes, \textit{operadores} de índice \textbf{[]}, informam ao compilador
que as variáveis \textbf{i} e \textbf{c} são do tipo array. Já o número
presente entre esses operadores define a quantidade de elementos a serem
alocados para o array em questão. No código acima, \textbf{i} é definido como
um array de \textit{int} com seis elementos e \textbf{c} como um array de
\textit{char} com três elementos.

\begin{center}
  \begin{bytefield}[endianness=little,bitwidth=6em]{6}
    i[6]\\
    \bitbox{1}{-57} \bitbox{1}{3} \bitbox{1}{791431480}
    \bitbox{1}{0} \bitbox{1}{32767} \bitbox{1}{0}\\
    \bitheader{0-5}\\
  \end{bytefield}
  \begin{bytefield}[endianness=little,bitwidth=2em]{3}
    c[3]\\
    \bitbox{1}{0} \bitbox{1}{-10} \bitbox{1}{49}\\
    \bitheader{0-2}
  \end{bytefield}
\end{center}

Uma boa prática de programação é inicializar os arrays, assim como os tipos
simples, pois, tanto em i[] quanto em c[], vemos claramente que os elementos
têm valores aleatórios. Isso se dá pelo lixo encontrado na área da memória
usada por esses recém-definidos arrays num escopo local.

Arrays também podem ser definidos com a quantidade de elementos informada por
outra variável, quando em escopo local. Mas atente sempre para inicializar a
variável que é responsável pela quantidade de elementos acima de zero, do
contrário comportamentos indesejados acontecerão, tais como pipocas voando na
tela ou suco de tangerina escorrendo pelos cantos do monitor.

\textit{Inicialize sempre a variável com a quantidade de elementos.}\\
\begin{ccode}
  int elements = 10;
  ...
  int i[elements];
\end{ccode}

\textit{Nunca se esqueça de inicializar! NUNCA! Ou o bicho vem pegar você...}\\

\subsection{Declarando Arrays}
Faz bem relembrarmos que declarar não é o mesmo que definir, portanto, ao
contrário do que se pensa comumente, arrays podem normalmente ser
declarados. Para tanto, basta fazermos uso do modificador \textit{extern} já
visto no item 3.3.

Forçando a quantidade de elementos \textbf{ext1[32]} ou não \textbf{ext2[]}.

\textit{array.h - Declaração}\\
\begin{ccode}
  extern int ext1[32];
  extern int ext2[];
\end{ccode}

E posteriormente definindo os mesmos arrays.

\textit{main.c - Definição}\\
\begin{ccode}
  #include "array.h"
  ...
  int ext1[32];
  int ext2[32];
  ...
\end{ccode}

\subsection{Inicializando Arrays}
Recordar é viver, então viva mais uma vez: \say{inicializar é atribuir valor no
  ato da declaração}. No caso de arrays podemos inicializá-los de várias
formas, vejamos.

Inicializando os elementos um a um.

\begin{ccode}
  int arr[3] = {1, 5, 10};
\end{ccode}

Inicializando os elementos um a um e forçando o compilador a reservar a
quantidade de elementos informados na inicialização. Neste caso, o array é
declarado e inicializado com 4 elementos.

\begin{ccode}
  int arr[] = {3, 2, 9, 1};
\end{ccode}

Inicializando todos os elementos com zero.

\begin{ccode}
  int arr[3] = {0};
\end{ccode}

Inicializando todos os elementos com zero (modo \say{preguiçoso suvina},
economiza um byte no código fonte por inicialização).

\begin{ccode}
  int arr[3] = {};
\end{ccode}

Inicializando o primeiro elemento com 5 e os demais com zero.

\begin{ccode}
  int arr[3] = {5};
\end{ccode}

Quando um array é definido num escopo global \textit{e/ou} com a classe de
armazenamento \textit{static} e não há inicialização explícita, de acordo com o
\textit{C Standard}, todos os elementos são inicializados com zero.

Inicializando todos os elementos com zero por definiçao estática.

\begin{ccode}
  static int array[6];
\end{ccode}

Inicializando por escopo global.

\begin{ccode}
  int arr[6];  /* Os elementos de arr iniciam
                  com zero, automaticamente.
                  "static" subentendido */

  int main(void)
  {
    return 0;
  }
\end{ccode}

Há, também, um array especial chamado array flexível. Ele só pode ser usado
como último membro de uma estrutura que já contém pelo menos outro membro. Esse
array se destina ao uso em estruturas com tamanho dinâmico que precisam ser
compartilhadas integralmente e/ou acessadas linearmente.

Veja abaixo exemplo de uso de um array flexível.

\begin{ccode}
  struct str {
    int body_len;
    double body[];
  };

  int length = sizeof(double [10]);

  struct str *s = malloc(sizeof(struct str) + length);
  s->body_len = length;
  ...
\end{ccode}

E há o array \say{sou brasileiro, não desisto nunca} que, tal qual o mote, não
tem sentido na vida real. Esse array aloca apenas um elemento.

\begin{ccode}
  int brazilian_i[] = {1};
  float brazilian_f[1] = {2};
\end{ccode}

\subsection{Acessando Arrays}
Os elementos dos arrays em C são acessados através do índice 0 até a quantidade
de elementos menos um (n-1), ou seja, num array com três elementos podemos
acessá-los da seguinte maneira.

\begin{ccode}
  float att[3] = {10.5, 25.3, 9.0};

  printf("%f\n", att[0]);
  printf("%f\n", att[1]);
  printf("%f\n", att[2]);
\end{ccode}

O código acima imprime na tela os valores dos elementos att[0] que é 10.5,
att[1] com 25.3, e o att[2] respectivamente 9.0. O array como qualquer variável
se encontra em locais específicos da memória. Antes do primeiro e logo após seu
último elemento há dados referentes a outros controles ou estruturas do próprio
programa, por isso, acessá-lo, inadvertidamente, além da área definida na
declaração/inicialização ocasionará seguramente comportamentos inesperados no
software (bugs), até porque o compilador não o avisará de plano sobre limites
extrapolados se você não estiver utilizando flags de warning (-Warray-bounds,
neste caso). Portanto, atenção redobrada e vergonha na cara.

\subsubsection{Atribuindo Valores a Arrays}
Até então vimos apenas como informar ao compilador o quanto de espaço
precisamos que ele reserve na memória para nossos dados e como atribuir valores
através da inicialização. \textit{Vamos bagunçar o coreto, então?}

\say{Pensei que não iria falar nada sobre atribuição...}

Foi mesmo, bateu uma preguiça... /o{\textbackslash} A verdade é que discorrer
brevemente o tema índices, também conhecido como subscritores, foi crucial para
mostrar a você, aspirante a programador C, que nesta bela linguagem, assim como
em grande parte das demais, o índice inicial de conjunto de dados é o 0 e não o
1. Sem esse conhecimento básico, talvez você já tivesse desistido de C e
iniciado a programar em Pascal.

\say{Mais falatório e não explicou como atribuo valores num array!}

Nervosinho, você, hein? Calma que é simples.

\begin{ccode}
  float att[3];

  att[0] = 1.1;
  att[1] = 1.2;
  att[2] = 1.3;
\end{ccode}

Pronto! Esse último código não inicializa o array deixando essa função para a
atribuição singular de cada elemento. Doeu tanto esperar?

\section{Array de Caracteres}
Este tipo de array costuma causar arrepios, mas ele, na realidade, não tem nada
de espetacular.

\say{E que arrepios são esses?}

Hmmm... Aquele arrepio de medo de coisas pipocarem na memória e possivelmente
no fluxo do programa. Mas calma, antes de você, ávido e inquieto leitor,
perguntar que pipocos são esses, respondo-lhe: portais do submundo por onde
\textit{hackers} do mal vêm à superfície calma do seu mundo virtual, caro
leitor. Mas isso é assunto para outro capítulo. Continuemos.

\say{Maas...}

Eu já disse rapá! Deixa de ser teimoso. Nunca ouviu falar que o apressado come
cru? Vôte!

Posso?

\say{...}

Vejamos. Um caractere tem uma representação visual cujo valor real está
atrelado a um standard de caracteres. O mais comum desses standards é o
\textit{\textbf{ASCII} - American Standard Code for Information Interchange}.

Um caractere na liguagem C deve ser representado entre duas aspas simples e é
chamado de \textit{one integer character constant}.

\begin{ccode}
  char c = 'A';
\end{ccode}

Acima vemos que \textit{c} é uma variável do tipo \textit{char} com o valor
\textit{A}, que não é nada mais que:

\begin{ccode}
  char c = 65;
\end{ccode}

Há também o \textit{multibyte integer character constant}.

\begin{ccode}
  int micc = 'Hi';
\end{ccode}

Cá entre nós, se não for realmente necessário, evite o uso dele uma vez que seu
valor dependerá da implementação definida (ambiente).

\say{Ok, anotado. Mas decimal sessenta e cinco? Como isso?}

Eu também ficava confuso nos meus primórdios informáticos, mas basta você
lembrar que, num computador, tudo são bits, estados 0 ou 1, que por sua vez,
aglutinam-se matematicamente e corroboram, numa concepção mais chula, com a
afirmação de que todos os dados são números. A esse respeito reveja os
primeiros capítulos, se houver alguma dúvida.

\say{-.-}

Esse valor 65 é o correspondente decimal ao caractere ASCII \textit{A}. Já a
conversão desse valor para a imagem bonita que conhecemos do caractere
\textit{A} se dá pelo processo de renderização por hardware/software o qual se
vale da própria tabela do standard envolvido. Mas voltemos ao C pois um só
caractere ASCII não faz palavra (em outros standards um só caractere poderá sim
significar palavra).

Conseguiremos esmiuçar melhor o que é um array de caracteres com o exemplo
abaixo.

\begin{ccode}
  char word1[] = {'A', 'u', 't', 'h', 'o', 'r', '\0'};
\end{ccode}

\say{Hmm... Eu só não entendi a barra invertida e o zero como último elemento do
  array.}

Esse elemento é necessário para construção de \textit{strings}. Em C não há um
tipo para strings como em outras linguagens, isso se dá por conta da
proximidade que C tem do hardware e pelo fato de na memória uma string não ser
nada mais que uma sequência de números que representam singularmente um
caractere em particular (ou parte dele em standards mais modernos). Assim esse
caractere é conhecido como o caractere NUL (de valor 0) na tabela ASCII e
significa, para qualquer código que interpreta strings nesse padrão, o fim do
texto a ser tratado.

O código acima poderia ser reescrito da seguinte forma sem prejuízo da
funcionalidade.

\begin{ccode}
  char word1[] = {'A', 'u', 't', 'h', 'o', 'r', 0};
\end{ccode}

Você verá em alguns códigos o uso de \textit{NUL} em vez do zero, mas não se
engane assim como eu me enganei por muito tempo, NUL não é uma keyword da
linguagem C, e sim apenas uma implementação acessória.

\begin{ccode}
  #define NUL '\0'
  /* or */
  #define NUL 0
  ...
  char word1[] = {'A', 'u', 't', 'h', 'o', 'r', NUL};
\end{ccode}

\say{Pelo que percebi, em C, usar um texto, melhor dizendo, uma string, é um
  baita pé no saco!}

Sim se o leitor adotar o método acima como regra. Em realidade, há outra forma
mais inteligente de se trabalhar com strings.

\begin{ccode}
  char word2[] = "Smug Author";
\end{ccode}

Basta inicializar o array com as famosas strings literais usando aspas duplas e
não mais a sintaxe de inicialização de elementos de arrays. E ainda dá para
usar concatenação ou mais de uma linha, se for necessário.

\begin{ccode}
  char word2[] = "Smug " "Author";
\end{ccode}

\begin{ccode}
  char word2[] = "Smug "
                 "Author";
\end{ccode}

\begin{ccode}
  char word2[] =
  "Smug \
  Author";
\end{ccode}

É saudável aconselhá-lo a nunca, NUNCA MESMO, definir o tamanho de um array
quando for usá-lo como string. Deixe esse serviço para o compilador, usando
apenas os colchetes vazios \textbf{[]}, pois ele calculará com exatidão o
tamanho a ser reservado na memória. Para mais sobre o tema, consulte a regra
\href{https://www.securecoding.cert.org/confluence/display/seccode/STR11-C.+Do+not+specify+the+bound+of+a+character+array+initialized+with+a+string+literal}{ARR02-C}
do CERT.

E aí o que achou?

\say{Gostei, mas você se esqueceu, nesses últimos exemplos, do caractere que
  finaliza a string. O zero.}

Olha que leitor astuto! Está de parabéns, contudo enganado. E a culpa é minha,
devo assumir, de tê-lo induzido até essa armadilha.

Há dois aspectos interessantes quando utilizamos strings literais:
\begin{enumerate}
\item O próprio compilador armazenará corretamente o caractere NUL após a
  string, e isso pode causar dores de cabeça aos desavisados que reservarem
  buffers menores do que o necessário.
\item Se estivermos usando \textit{ponteiros} o texto passa a ser armazenado no
  segmento \textit{read-only-data}, portanto, dependendo da arquitetura, não
  poderá ser alterado.
\end{enumerate}

Sem desmerecer o primeiro aspecto, ressalto que o segundo é bem capcioso, uma
vez que algumas arquiteturas podem não suportar segmentos \textit{read-only},
ou seja, permitirão a alteração dessa região da memória mesmo inicializada
através de ponteiro. Em suma, mais um comportamento indefinido da linguagem.

\say{Curioso... E o que são esses ponteiros mesmo?}

Calma jovem padawan, o capiroto o aguarda faz tempo e não se reclama disso.

\say{Vôte!}


\section{Arrays na Vida Real?!}
\subsection{Imprimir Elementos}
\subsection{Somar Elementos}
\subsection{Inverter o Array}
\subsection{Ordenar o Array}

\section{Arrays com Parâmetros}

\section{Arrays Multidimensionais (Matrizes)}


%------------------------------------------------
