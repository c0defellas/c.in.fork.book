\chapterimage{chapter_pointers.jpg} % Chapter heading image

\chapter{Vetores (Arrays)}
Os vetores (arrays), em C, são tipos de dados de alocação contíngua derivados dos tipos simples. Em outras palavras, são estruturas de dados homogêneas compostas de elementos de mesmo tipo e posicionados em sequência na memória. Segue exemplo de vetor com seis elementos:

\begin{center}
  \begin{bytefield}[endianness=little,bitwidth=6em]{6}
    \bitbox{1}{9} \bitbox{1}{20} \bitbox{1}{11}
    \bitbox{1}{7} \bitbox{1}{43} \bitbox{1}{31}\\
    \bitheader{0-5}
  \end{bytefield}
\end{center}
\section{Trabalhando com Vetores}
\subsection{Definindo Vetores}

Ao se definir um vetor é necessário determinar o respectivo tamanho.

\begin{ccode}
  int i[6];
  char c[3];
\end{ccode}

O uso do operador de índice [] é que informa ao compilador que as variáveis i e c são do tipo vetor. Já o número entre os colchetes define a quantidade de elementos a serem alocados para o vetor em questão. No código acima, i é definido como um vetor de int com seis elementos e c um vetor de char com três elementos.

\begin{center}
  \begin{bytefield}[endianness=little,bitwidth=6em]{6}
    i[6]\\
    \bitbox{1}{-57} \bitbox{1}{3} \bitbox{1}{791431480}
    \bitbox{1}{0} \bitbox{1}{32767} \bitbox{1}{0}\\
    \bitheader{0-5}\\
  \end{bytefield}
  \begin{bytefield}[endianness=little,bitwidth=2em]{3}
    c[3]\\
    \bitbox{1}{0} \bitbox{1}{-10} \bitbox{1}{49}\\
    \bitheader{0-2}
  \end{bytefield}
\end{center}

Uma boa prática de programação é inicializr os vetores, assim como os tipos simples, pois, tanto em i[6] quanto em c[3], vê-se claramente que os elementos contêm valores aleatórios.

\subsection{Declarando Vetores}

Ao contrário do que se pensa comumente, vetores podem ser declarados sem especificação da quantidade de elementos.

\begin{ccode}
  extern int ext[];
\end{ccode}

\subsection{Inicializando Vetores}

\begin{ccode}
  static int stc[6];
\end{ccode}

\subsection{Acessando Vetores}

\section{Vetor de Caracteres}

\section{Usando Vetores}
\subsection{Imprimir Elementos}
\subsection{Somar Elementos}
\subsection{Inverter o Vetor}
\subsection{Ordenar o Vetor}

\section{Vetores Multidimensionais}
Conversao.

\section{Vetores como Parâmetros}

%------------------------------------------------
