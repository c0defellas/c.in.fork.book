%================================================
% CAPÍTULO ARRAYS
%================================================

\chapterimage{chapter_pointers.jpg} % Chapter heading image

\chapter{Arrays}
Quem nunca precisou organizar dados de tipo semelhante num único local? Eu sim, você não? Então quando você, por exemplo, escreve um texto, o que acha que está fazendo? \say{\textit{Organizando ideias?}}. Sim, também! Contudo, o que você está organizando materialmente são caracteres.

Imagine agora termos que, para cada caractere organizado, definir uma variável do tipo \textit{char} e atribuir-lhe o valor respectivo.

\begin{ccode}
  char c1, c2, c3, c4, c5, c6, c7, c8, c9, c10,
       c11, c12, c13, c14, c15, c16, c17;

  c1 = 'A';
  c1 = 'u';
  c1 = 't';
  c1 = 'o';
  c1 = 'r';
  c1 = ' ';
  ...
\end{ccode}
\\

Tal exemplo é deveras trabalho do capiroto, e só será utilizado na vida real se o programador tiver algum pacto com esse tinhoso. Para os que seguem o caminho da luz e dos ensinamentos jedi, há uma alternativa feliz.

\textbf{Arrays}, em C, são tipos de dados de alocação contíngua derivados dos tipos simples. Em outras palavras, são estruturas de dados homogêneas, compostas de elementos de mesmo tipo, e posicionados em sequência na memória. Arrays também são conhecidos como \textit{vetores}, quando a sequência dos dados é unidimensional (linear), e como \textit{matrizes}, quando essa sequência se dá multimensionalmente (linhas e colunas). Divagaremos a respeito dos arrays do tipo matriz mais a frente, abordando, por enquanto, o array do tipo vetor. Seguem dois exemplos de arrays.

\begin {center}
  \begin{bytefield}[bitwidth=1em]{17}
    \bitbox{17}{Autor presunçoso!}\\
    \bitbox{1}{A} \bitbox{1}{u} \bitbox{1}{t} \bitbox{1}{o} \bitbox{1}{r} \bitbox{1}{ }
    \bitbox{1}{p} \bitbox{1}{r} \bitbox{1}{e} \bitbox{1}{s} \bitbox{1}{u} \bitbox{1}{n}
    \bitbox{1}{ç} \bitbox{1}{o} \bitbox{1}{s} \bitbox{1}{o} \bitbox{1}{!}
  \end{bytefield}
\end{center}

A frase \say{Autor presunçoso!} é a organização de dezessete caracteres um atrás do outro. Isso mesmo, dezessete, pois o espaço também é um dado. Nunca se esqueça que, quando tratamos dados, até um \say{0} ocupa espaço na memória do computador.

Segue outra organização de dados. Desta vez de inteiros que serão o resultado do próximo sorteio da mega-sena. Duvida?

\begin{center}
  \begin{bytefield}[endianness=little,bitwidth=6em]{6}
    \bitbox{1}{9} \bitbox{1}{20} \bitbox{1}{11}
    \bitbox{1}{7} \bitbox{1}{43} \bitbox{1}{31}\\
    % \bitheader{0-5}
  \end{bytefield}
\end{center}

\begin{remark}
Em algumas fontes esparsas de idioma português, você, leitor, encontrará referências a arrays como sendo \textit{\textbf{arranjos}}. Ressaltamos que um array, numa tradução mais fiel, é um \textit{\textbf{conjunto}} de dados. Já um arranjo:
  \begin{figure}[!htp]
    \centering
    \includegraphics[scale=0.8, keepaspectratio=true]{arrays_arrange.jpg}
    \caption{Array não é Arranjo}
    \label{fig:arrays_arrange}
  \end{figure}
\end{remark}

\section{Trabalhando com Arrays}
\subsection{Definindo Arrays}

Ao definirmos um array, precisamos determinar a respectiva quantidade de elementos.

\begin{ccode}
  int i[6];
  char c[3];
\end{ccode}
\\

Os colchetes, \textit{operadores} de índice \textbf{[]}, informam ao compilador que as variáveis \textbf{i} e \textbf{c} são do tipo array. Já o número presente entre esses operadores define a quantidade de elementos a serem alocados para o array em questão. No código acima, \textbf{i} é definido como um array de \textit{int} com seis elementos e \textbf{c} como um array de \textit{char} com três elementos.

\begin{center}
  \begin{bytefield}[endianness=little,bitwidth=6em]{6}
    i[6]\\
    \bitbox{1}{-57} \bitbox{1}{3} \bitbox{1}{791431480}
    \bitbox{1}{0} \bitbox{1}{32767} \bitbox{1}{0}\\
    \bitheader{0-5}\\
  \end{bytefield}
  \begin{bytefield}[endianness=little,bitwidth=2em]{3}
    c[3]\\
    \bitbox{1}{0} \bitbox{1}{-10} \bitbox{1}{49}\\
    \bitheader{0-2}
  \end{bytefield}
\end{center}

Uma boa prática de programação é inicializar os arrays, assim como os tipos simples, pois, tanto em i[] quanto em c[], vemos claramente que os elementos têm valores aleatórios. Isso se dá pelo lixo encontrado na área da memória usada por esses recém-definidos arrays num escopo local.

Arrays também podem ser definidos com a quantidade de elementos informada por outra variável, quando em escopo local. Mas atente sempre para inicializar a variável que é responsável pela quantidade de elementos acima de zero, do contrário comportamentos indesejados acontecerão, tais como pipocas voando na tela ou suco de tangerina escorrendo pelos cantos do monitor.

\textit{Inicialize sempre a variável com a quantidade de elementos.}\\
\begin{ccode}
  int elements = 10;
  ...
  int i[elements];
\end{ccode}
\\

\textit{Nunca esqueça de inicializar! NUNCA! Ou o bicho vem te pegar...}\\

\subsection{Declarando Arrays}
Faz bem relembrarmos que declarar não é o mesmo que definir, portanto, ao contrário do que se pensa comumente, arrays podem normalmente ser declarados. Para tanto, basta fazermos uso do modificador \textit{extern} já visto no item 3.3.

Forçando a quantidade de elementos \textbf{ext1[32]} ou não \textbf{ext2[]}.

\textit{array.h - Declaração}\\
\begin{ccode}
extern int ext1[32];
extern int ext2[];
\end{ccode}
\\

\textit{main.c - Definição}\\
\begin{ccode}
#include ``array.h''
...
int ext1[32];
int ext2[32];
...
\end{ccode}


\subsection{Inicializando Arrays}
Recordar é viver, então viva mais uma vez: \say{inicializar é atribuir valor no ato da declaração}. No caso de arrays podemos inicializá-los de várias formas, vejamos.
\\

Inicializando os elementos um a um.

\begin{ccode}
  int arr[3] = {1, 5, 10};
\end{ccode}
\\

Inicializando os elementos um a um e forçando o compilador a reservar a quantidade de elementos informados na inicialização. Neste caso, o array é declarado e inicializado com 4 elementos.

\begin{ccode}
  int arr[] = {3, 2, 9, 1};
\end{ccode}
\\

Inicializando todos os elementos com zero.

\begin{ccode}
  int arr[3] = {0};
\end{ccode}
\\

Inicializando todos os elementos com zero (modo \say{preguiçoso suvina}, economiza um byte no código fonte por inicialização).

\begin{ccode}
  int arr[3] = {};
\end{ccode}
\\

Inicializando o primeiro elemento com 5 e os demais com zero.

\begin{ccode}
  int arr[3] = {5};
\end{ccode}
\\

Quando um array é definido num escopo global \textit{e/ou} com a classe de armazenamento \textit{static} e não há inicialização explícita, de acordo com o \textit{C Standard}, todos os elementos são inicializados com zero.

Inicializando todos os elementos com zero por definiçao estática.

\begin{ccode}
  static int array[6];
\end{ccode}
\\

Inicializando por escopo global.

\begin{ccode}
int arr[6];    /* Os elementos de arr são inicializados
                  com zero, automaticamente.
                  ``static'' implícito */

int main(void)
{
  return 0;
}
\end{ccode}
\\

Há, também, umas inicializações curiosas que não geram erro na compilação mas nunca devem ser utilizados por suas desonesta representação, referência errônea aos valores atribuídos e total falta de sentido. Vejamos.

Abaixo dois arrays \say{fantasmas} de tamanho 0 que assombram até ateu. \textit{São fantasmas de variáveis do passado.}

\begin{ccode}
  int haunts_even_atheists1[] = {};
  int haunts_even_atheists2[0] = {};
\end{ccode}
\\

E há o array \say{sou brasileiro, não desisto nunca} que, tal qual o mote, não tem sentido na vida real. Esse array aloca apenas um elemento.

\begin{ccode}
  int brazilian_i[] = {1};
  float brazilian_f[1] = {2};
\end{ccode}
\\

\subsection{Atribuindo Valores em Arrays}
Aqui falar sobre índice e demonstrar como fica o array na memória após a atribuição.

\begin{ccode}
  float att[2];
  att[0] = 10.5;
  att[1] = 25.3;
\end{ccode}

\subsection{Acessando Arrays}
Discorrer brevemente sobre o acesso aos elementos de um array.

\section{Array de Caracteres}

\section{Arrays na Vida Real?!}
\subsection{Imprimir Elementos}
\subsection{Somar Elementos}
\subsection{Inverter o Array}
\subsection{Ordenar o Array}

\section{Arrays como Parâmetros}

\section{Arrays Multidimensionais (Matrizes)}


%------------------------------------------------
