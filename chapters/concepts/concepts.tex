% !TEX encoding = UTF-8 Unicode
%================================================
% CAPÍTULO CONCEITOS
%================================================

\chapterimage{chapter_pointers.jpg} % Chapter heading image

\chapter{Conceitos fundamentais}
Antes de começarmos a falar sobre C devemos nos perguntar: o que é o C?
\begin{quote}
\textit{C é uma linguagem de programação, cara!}
\end{quote}

Ok, ok. Você está certo. Mas entender o que é uma linguagem de programação é o primeiro passo. Se você já sabe o que é uma, não tenha pressa. Quem sabe você aprenda algo novo ou melhore seu conceito. Ou quem sabe contribua com este livro e melhore esta seção.

Bem, uma linguagem de programação é uma forma estruturada (e organizada) criada com a intenção de comunicar a uma máquina que ela deve realizar certas instruções e comandos. Esta máquina, não por acaso, é um computador (pelo menos no nosso caso).

A forma que estas instruções e comandos são ordenados irá alterar o que o computador entenderá e consequentemente o que ele executará. Este passo-a-passo é conhecido como \textit{algoritmo}.



\section{Linguagens de programação}

Linguagens de programação.

%------------------------------------------------

\section{Sistemas operacionais}

Linux.

%================================================

\section{O que são compiladores}

Compiladores.

%================================================

\section{Ambientes de desenvolvimento}

Ambientes de desenvolvimento.


