% !TEX encoding = UTF-8 Unicode
%================================================
% CAPÍTULO CONCEITOS
%================================================

\chapterimage{chapter_pointers.jpg} % Chapter heading image

\chapter{Conceitos fundamentais}
Antes de começarmos a falar sobre C devemos nos perguntar: o que é o C?
\begin{quote}
\textit{C é uma linguagem de programação, cara!}
\end{quote}

Ok, ok. Você está certo. Mas entender o que é uma linguagem de programação é o primeiro passo. Se você já sabe o que é uma, não tenha pressa. Quem sabe você aprenda algo novo ou melhore seu conceito. Ou quem sabe contribua com este livro e melhore esta seção.

Bem, uma linguagem de programação é uma forma estruturada (e organizada) criada com a intenção de comunicar a uma máquina que ela deve realizar certas instruções e comandos. Esta máquina, não por acaso, é um computador (pelo menos no nosso caso).

A forma que estas instruções e comandos são ordenados irá alterar o que o computador entenderá e consequentemente o que ele executará. Este passo-a-passo é conhecido como \textit{algoritmo}.

Porém, como nem tudo são flores, existem várias maneiras de se passar estas instruções para o computador. Provavelmente não existe limite para o desejo humano de criar coisas novas; e assim uma nova linguagem é criada com mais frequência do que se ganha na loteria. Até o Google já estudou a possibilidade de criar a sua!

\begin{quote}
\textit{E o quê eu ganho com isso? Só mais uma linguagem para aprender?}
\end{quote}

Bom..você não precisa aprender todas, correto? A não ser que realmente queira, não recomendo isto!

Então é preciso escolher a linguagem (ou algumas delas) que atenda esse seu desejo "estranho" de programar. Como este é um livro sobre linguagem C, não preciso nem dizer que você começou muito bem, preciso? ;)

A seguir, vamos falar um pouco mais sobre algumas categorias de linguagens e suas diferenças.

\section{Linguagens de programação}

\subsection{Assembly}
É considerada uma linguagem de máquina. Suas instruções são o que há de mais próximo da linguagem da máquina mas ainda assim legível pelos humanos (a não ser que você seja o Neo da Matrix). Suas instruções estão intimamente relacionadas com algo chamado de "Conjunto de instruções" (Instruction Set Architecture, em inglês).

\subsection{CLI}
Apesar de não serem comumente encaradas como linguagens de programação, as "linguagens de linha de comando" (Command Line Interface< em inglês) são linguagens bastante utilizadas. Quer apostar?

Quem já trabalhou com sistemas operacionais Unix, Linux e Windows com certeza já viu estas linguagens: Windows batch (aquela tela preta do command/cmd), Windows PowerShell (administradores de Windows Server conhecem bem), bash, sh, zsh, ksh (todas presente em sistemas Unix e Linux).

\subsection{Linguagens compiladas}
É aqui que o C cai! Uma linguagem compilada!
"Compilada? Explica melhor.". Bom, elas passam por compiladores. "Não me diga!".
Teoricamente, qualquer linguagem pode ser do tipo compilada (ou interpretada). A diferença é que uma linguagem que está sendo compilada é traduzida do seu código fonte para instruções de máquina quase que diretamente. E este é um assunto polêmico.

A princípio, linguagens de baixo nível (mais próximas da máquina) são compiladas, pois se busca uma maior "eficiência". Como o código é traduzido completamente antes de ser exposto à máquina, não há necessidade de traduzir parte do código no momento execução. Por outro lado, se desejarmos executar este código em uma máquina que possui uma arquitetura diferente, teremos que realizar uma nova tradução (no nosso caso, uma nova compilação).



%------------------------------------------------

\section{Sistemas operacionais}

Linux.

%================================================

\section{O que são compiladores}

Compiladores.

%================================================

\section{Ambientes de desenvolvimento}

Ambientes de desenvolvimento.


